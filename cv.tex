%%%%%%%%%%%%%%%%%%%%%%%%%%%%%%%%%%%%%%%%%
% Medium Length Professional CV
% LaTeX Template
% Version 2.0 (8/5/13)
%
% This template has been downloaded from:
% http://www.LaTeXTemplates.com
%
% Original author:
% Trey Hunner (http://www.treyhunner.com/)
%
% Important note:
% This template requires the resume.cls file to be in the same directory as the
% .tex file. The resume.cls file provides the resume style used for structuring the
% document.
%
%%%%%%%%%%%%%%%%%%%%%%%%%%%%%%%%%%%%%%%%%

%----------------------------------------------------------------------------------------
%	PACKAGES AND OTHER DOCUMENT CONFIGURATIONS
%----------------------------------------------------------------------------------------

\documentclass{resume} % Use the custom resume.cls style

\usepackage[left=0.75in,top=0.6in,right=0.75in,bottom=0.6in]{geometry} % Document margins

\name{Sheng Yang} % Your name
%\address{Via valeggio 5 \\ Padova, Italy 35125} % Your address
%\address{(+39)~$\cdot$~3891438661 \\ sheng.yang@oapd.inaf.it} % Your phone number and email

\address{Dept. of Physics, University of California Davis, 1 Shields Ave, Davis, CA 95616} % Your address
\address{(+1)~$\cdot$~520-229-1170 \\ sheng.yang@oapd.inaf.it/sngyang@ucdavis.edu} % Your phone number and email


\begin{document}

%----------------------------------------------------------------------------------------
%	EDUCATION SECTION
%----------------------------------------------------------------------------------------

\begin{rSection}{Education}
{University of Padova, Padova, Italy} \\
{PHD, Department of Astronomy} \hfill 10/2015 - Present\\

{Beijing Normal University, Beijing, China} \\
{M.S., Department of Astronomy} \hfill 09/2012 - 06/2015\\

{Central China Normal University, Wuhan, China }\\
{B.S., Physics}  \hfill 09/2008 - 06/2012\\

\end{rSection}

%----------------------------------------------------------------------------------------
%	WORK EXPERIENCE SECTION
%----------------------------------------------------------------------------------------

\begin{rSection}{Research Experience}
       {\textbf{Department of Physics, University of California Davis, CA, US}}  \hfill 10/2016 - Present\\
       {\textbf{Research Topics}}
         \begin{itemize}  \itemsep -2pt %reduce space between items
         \item \textit{Searching for the electromagnetic counterpart of gravitational wave sources - based on dlt40}
         \end{itemize}

       {\textbf{Observatory of Padova, Padova, Italy}}  \hfill 10/2015 - Present\\
       {\textbf{Research Topics}}
         \begin{itemize}  \itemsep -2pt %reduce space between items
         \item \textit{Searching for the electromagnetic counterpart of gravitational wave sources - based on grawita}
         \end{itemize}

       {\textbf{Institute of High Energy Physics, \\Chinese Academy of Science, Beijing, China}} \hfill 10/2013 - 10/2015\\
       {\textbf{Research Topics}}
         \begin{itemize}  \itemsep -2pt %reduce space between items
         \item \textit{Remove the systematics for large scale structure survey and the improvement of restrictions on PNG}
         \item \textit{Constraining neutrino properties with a Euclid-like galaxy cluster survey}
         \end{itemize}

       {\textbf{Central China Normal University, Wuhan, China}} \hfill 09/2008 - 06/2012\\
        {\textbf{Research Topics}}
                 \begin{itemize}  \itemsep -2pt %reduce space between items
                \item \textit{The influence of vibration on the resonant cavity mirror of the particle movement}        
                \item \textit{Infrared diode laser wavelength calibration}              
                \item \textit{The dynamic evolution of newly born millisecond magnetar wind}
                \end{itemize}                
                
       {\textbf{Institute of Physics and Mathematics, \\Chinese Academy of Science, Wuhan, China}} \hfill 06/2011 - 10/2011\\
        {\textbf{Research Topics}}
                 \begin{itemize}  \itemsep -2pt %reduce space between items
             \item \textit{Laser frequency stabilization experiment based on LabVIEW}    
        \end{itemize}         
                        
                      
\end{rSection}

\begin{rSection}{Technical Strengths}

\begin{tabular}{ @{} >{\bfseries}l @{\hspace{6ex}} l }
Computer Languages & Python, Idl, Fortran, C, Labview \\
Programming software &  Matlab, Mathematica\\
Other tools & iraf, sextractor, hotpants, latex,  Healpix.
\end{tabular}

\end{rSection}

%----------------------------------------------------------------------------------------
%	EXAMPLE SECTION
%----------------------------------------------------------------------------------------

%\begin{rSection}{Section Name}

%Section content\ldots

%\end{rSection}

%----------------------------------------------------------------------------------------

\end{document}
